\def\changemargin#1#2{\list{}{\rightmargin#2\leftmargin#1}\item[]}
\let\endchangemargin=\endlist 

\begin{changemargin}{0.75in}{0.75in}

\section*{Preface}

This book contains the result of the collective efforts, gathered
knowledge, and engineered algorithms and implementations of the
fourteen students in my Fall 2024 senior project course: CSC-489
\emph{Guided Research in Computer Science}.

The students worked to understand and offer solutions to a
challenging, outstanding open problem in theoretical computer science:
\begin{quote}
\begin{center}
    \emph{How difficult is the Maximum Matching Problem?}
\end{center}
\end{quote}
Each student charted their own path to explore this question.  Some
dove head first into implementation of algorithms.  Some, more
circumspectly, investigate the deep literature on this problem and
computational complexity theory more broadly.  These initial forays
into the wild frontiers of human knowledge, lead to new questions,
techniques, understanding, and connections with other problems.

On its surface this problem seems simple, but as the investigation
deepened they learned that only some cases of the problem are known to
have efficient, polynomial-time algorithms, and some variants are
NP-complete, making it unlikely there will be an efficient algorithm
that can solve it exactly.  This dichotomy lead some students to
investigate the space of efficient algorithms and some to study
related NP-complete problems and ways to solve them.

Some students took paths that I had foreseen into classical matching
algorithms, perfect matching algorithms, stable marriage problems, and
linear and integer programming, but some students delved into rich,
connected topics that I had not, like ant colony optimization,
multi-threading, and real-world applications of matching, like schemes
for organizing kidney donations.

Some of these paths converged quickly, leading to collaborative
efforts, among them developing a testing and evaluation framework for
a problem we do not how to efficiently solve.  Some paths saw students
to take on leadership roles in the research, implementation, and
writing efforts, charting the path for themselves, but also guiding
and supporting their peers.

In the end, all paths taken converge here in this book, which is a
record of the knowledge gained, and our successes and failures in
attempting to answer this vexing question.

I am proud of the work my students accomplished.  It has been a
pleasure watching them grow into mature, professional, reflective,
computer scientists over the last four years---I look forward to
seeing the paths they embark on after Union!

\bigskip

\noindent Schenectady, November 2024 \hfill Matthew Anderson

\end{changemargin}
