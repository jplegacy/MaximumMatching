% Author @ Kurtik Appadoo, IN-PROGRESS, Reviewer @ Janak Subedi, Due Date 11/11
\subsection{Automating Test Generation}
Automating test generation offers significant advantages and presents notable challenges that must be managed to ensure reliable outcomes. Automated test generation can greatly reduce the manual time and effort required to create comprehensive test suites. By automating this process, developers can test a wide range of input scenarios, resulting in more robust and reliable software. This is especially beneficial for large-scale projects, where maintaining extensive test coverage manually is time-consuming and prone to error. Automated tests can be run consistently and repeatedly, enabling early and frequent detection of issues, thus maintaining the integrity of the codebase over time.

However, challenges arise when automating test generation for complex problem domains, such as NP-Complete problems. Generating test cases for d-partite graphs with known solutions is particularly difficult due to their computational intensity. Solving these problems often requires exhaustive search algorithms that check all possible combinations, which is resource-intensive and not easily automated without substantial computational power. The processing time and resources needed for generating and verifying such test cases make automation demanding.

Verification introduces additional complexity. Automating the verification process requires a reliable, efficient algorithm to confirm the correctness of solutions. This raises an important question: how can we ensure that the verification algorithm is accurate? This can create a cycle where the verification algorithm itself needs testing. To address this, verification is typically performed using well-established, peer-reviewed algorithms. Alternatively, a backtracking algorithm can be employed, as it guarantees correctness by exploring all possible solutions to confirm the maximum matching.

Another challenge in automated test generation is the potential bias in algorithmically generated test cases. While these instances may appear random, they are produced by human-written algorithms, which can introduce unintended patterns. This lack of complete randomness can affect the thoroughness and effectiveness of tests. For example, when generating d-partite graphs for perfect matching, it was noted that while hyperedges were generated randomly, the initial perfect matching structure often followed a pattern. This predictability posed a risk: solvers optimized to recognize such patterns might show misleading results in terms of speed and accuracy.

Addressing these issues requires introducing more randomness and variability into the generation process to create diverse test suites. Until true randomness is achieved, specialized solvers, particularly those using backtracking algorithms that exploit specific structures, may yield results that do not reflect their performance in general cases. Continuous refinement of automated test generation tools is needed to ensure they remain unbiased and representative.