% Author: Ryan Feneley, IN-PROGRESS, Reviewer: Ethan K, Due Date: 11/19
\subsection*{Auction Algorithm (1981)}

The Auction Algorithm, introduced by Dimitri Bertsekas in 1981, is an iterative method designed to solve the maximum weighted matching problem. Inspired by the dynamics of auctions, the algorithm operates by associating prices with vertices and iteratively refining these prices to maximize the total weight of the matching \cite{bertsekas1981auction}. It is particularly effective for solving large-scale assignment problems and has found applications in various domains due to its scalability and simplicity.

\paragraph{Algorithm Mechanism}
The Auction Algorithm relies on a bidding and assignment process to allocate vertices optimally. Its key steps are:
\begin{enumerate}
    \item Initialization: Each vertex in one partition of the graph is initially unassigned, and all prices are set to zero.
    \item Bidding Phase: Each unassigned vertex determines the edge with the highest "profit," which is defined as the weight of the edge minus the price of the connected vertex in the other partition. The vertex bids an increment over the current highest price to secure this edge.
    \item Assignment Phase: Vertices in the other partition accept the highest bid, updating their assignments and prices accordingly.
    \item Iterative Refinement: The process repeats until all vertices are assigned or a predefined termination criterion is met.
\end{enumerate}

This auction-based mechanism ensures that vertices compete for assignment, driving the solution toward an optimal matching.

\paragraph{Performance and Complexity}
The Auction Algorithm achieves a time complexity of \(O(n^2 \log n)\), where \(n\) is the number of vertices in the graph. This performance makes it highly efficient for dense graphs and large-scale problems. Furthermore, its decentralized nature allows for parallelization, which can significantly speed up computations in distributed systems.


\paragraph{Strengths and Limitations}
The primary strength of the Auction Algorithm lies in its simplicity and scalability. Its iterative approach is well-suited to parallel computation, making it efficient for large datasets. However, the algorithm may face challenges in sparse graphs where competition for edges is limited, leading to slower convergence. Additionally, while the algorithm guarantees optimal solutions in weighted bipartite graphs, its extension to non-bipartite settings requires significant adaptation.

\paragraph{Conclusion}
The Auction Algorithm remains a cornerstone in the study of maximum weighted matching problems. Its ability to balance computational efficiency with solution quality has made it a staple in both theoretical research and practical applications. By leveraging market-inspired dynamics, it offers a unique approach to optimization that continues to influence the development of advanced algorithms in combinatorial optimization.
