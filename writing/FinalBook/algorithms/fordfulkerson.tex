% Author @ Vinh, IN-PROGRESS, Reviewer @ Kurtik, Due Date 11/11

\subsubsection*{Ford-Fulkerson Algorithm (1956)}

The Ford-Fulkerson Algorithm, introduced by L.R. Ford Jr. and D.R. Fulkerson in 1956, is a pivotal method for computing the maximum flow in a flow network, which can be applied to solve the maximum matching problem in bipartite graphs. Although originally designed for general flow networks, the algorithm's ability to find augmenting paths makes it suitable for matching problems by transforming the bipartite graph into a network flow model \cite{ford_fulkerson}.

\textbf{Algorithm Overview}

The Ford-Fulkerson method operates by repeatedly finding augmenting paths from the source to the sink in the residual graph and increasing the flow until no more augmenting paths exist. The algorithm starts with zero flow in all edges. While there exists a path from the source to the sink in the residual graph, the algorithm determines the minimum residual capacity along the path, increases the flow along the path by this minimum capacity, and adjusts the residual capacities of the edges and reverse edges accordingly.

In the context of bipartite graphs, the vertices can be divided into a source, a sink, and two partitions representing the bipartition. Edges are directed from the source to one partition, then to the other partition, and finally to the sink. This allows the algorithm to find a maximum matching by treating it as a maximum flow problem \cite{cormen}.

\textbf{Performance and Time Complexity}

The time complexity of the Ford-Fulkerson algorithm depends on the method used to find augmenting paths. If the paths are chosen arbitrarily, the algorithm may run indefinitely in the case of irrational capacities. However, when capacities are integers and a breadth-first search is used to find the shortest augmenting paths (as in the Edmonds-Karp implementation), the algorithm runs in $O(V E^2)$ time, where $V$ is the number of vertices and $E$ is the number of edges~\cite{ford_fulkerson}.

Compared to specialized algorithms like Hopcroft-Karp, which runs in $O(\sqrt{V} E)$ time for bipartite graphs, the Ford-Fulkerson method is less efficient for large graphs. Nevertheless, its simplicity and general applicability make it a fundamental algorithm in network flow theory.

\textbf{Applications and Limitations}

The Ford-Fulkerson algorithm is widely used in various applications such as network routing, image segmentation, and scheduling problems. It can be used to solve a broad range of maximum flow and matching problems \cite{ahuja}.

However, the algorithm has limitations. For large bipartite graphs, it is inefficient compared to more advanced algorithms like Hopcroft-Karp, which have better time complexity. Additionally, the total number of iterations can be significantly affected by the choice of augmenting paths; without careful selection, the algorithm may perform poorly. Furthermore, if the graph has irrational edge capacities, the algorithm may not terminate, as the flow can increase indefinitely in smaller and smaller increments.

Overall, Ford-Fulkerson algorithm is essential in the study of network flows and is a cornerstone in developing more advanced algorithms.

