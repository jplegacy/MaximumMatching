% Author: Ryan Feneley, Reviewer: Ethan K, Due Date: 11/19
The Greedy Matching Algorithm is a simple yet effective heuristic for finding large matchings in graphs. While it does not guarantee a maximum matching, it is widely used in practice due to its low computational complexity and ease of implementation \cite{pettie2004greedy}.

\paragraph{Algorithm Overview}
The Greedy Matching Algorithm iterates over the edges of the graph, selecting edges to include in the matching as long as they do not share vertices with edges already in the matching. The algorithm proceeds as follows \cite{avis1983greedy}:
\begin{itemize}
    \item \textbf{Step 1:} Start with an empty matching.
    \item \textbf{Step 2:} Sort the edges in some order (usually arbitrary, although edge weights or degrees can guide the choice).
    \item \textbf{Step 3:} For each edge in the sorted list, add it to the matching if neither of its endpoints is already matched.
    \item \textbf{Step 4:} Repeat until all edges have been processed.
\end{itemize}

\paragraph{Performance and Time Complexity}
The Greedy Matching Algorithm has a time complexity of \( O(E \log E) \), where \(E\) is the number of edges in the graph \cite{pettie2004greedy}. In practice, it often runs faster due to the simple decision-making process at each step. Despite its simplicity, the greedy approach can produce a matching that is close to the maximum in many types of graphs, particularly sparse graphs or graphs with random structures \cite{benson2018tutorial}.

\paragraph{Applications and Limitations}
This algorithm is commonly used in applications where computational resources are limited, and finding an exact or near-optimal matching quickly is more important than guaranteeing a maximum matching. Examples include real-time scheduling, online bipartite matching, and network design problems \cite{avis1983greedy}.

However, the algorithm’s main limitation is that it can produce suboptimal results, especially in dense graphs where it may miss augmenting paths that could lead to a larger matching. For this reason, it is often used as a starting point or in combination with other more sophisticated algorithms \cite{pettie2004greedy}.
