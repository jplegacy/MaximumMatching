% Author: Ryan Feneley, Reviewer: Ethan K, Due Date: 11/19
Despite significant advancements in algorithms for the maximum matching problem, various limitations continue to challenge their practical applicability and performance in diverse scenarios.

\subsection{Scalability}
Many exact algorithms for maximum matching suffer from high computational complexity, limiting their feasibility for large-scale graphs. For example, the Hungarian algorithm \cite{kuhn1955hungarian} exhibits \(O(n^3)\) time complexity, which becomes prohibitive in large applications. Although more recent algorithms, such as the Hopcroft-Karp algorithm \cite{hopcroft1973n} and Gabow's efficient implementation of Edmonds' algorithm \cite{gabow1976efficient}, offer improved complexities in specific cases (e.g., \(O(\sqrt{V}E)\) for bipartite graphs), they are still constrained in scalability when processing very large datasets or graphs with high densities. Moreover, the advent of matrix multiplication techniques has allowed faster exact solutions in certain bipartite cases \cite{karp1990deterministic}; however, such approaches are still limited by the overhead associated with matrix operations, as discussed in Karp et al. (1990) \cite{karp1990deterministic}.

\subsection{Optimality}
While heuristic and approximation algorithms can provide solutions faster than exact methods, they do not always guarantee optimal matchings. The Karp-Sipser algorithm, for instance, performs exceptionally well in sparse random graphs \cite{aronson1997maximum}, yet it can yield suboptimal results in dense graphs where pendant vertices are rare. Additionally, randomized heuristics, such as those proposed by Karp and Lovász \cite{karp2011maximum}, are effective for approximations but carry a probability of deviation from the true maximum matching. In applications such as network design and scheduling, where match quality directly impacts performance, these limitations can be critical. The inability to guarantee optimality necessitates careful consideration of the trade-offs involved when selecting non-exact algorithms \cite{lovasz2013randomized}.

\subsection{Complexity in Special Cases}
Algorithms often face challenges when dealing with specific graph structures. For instance, exact algorithms designed for bipartite graphs, such as the Ford-Fulkerson method \cite{ford1956flows}, may underperform on general or directed graphs due to structural mismatches. As Lovász and Szegedy (2013) \cite{lovasz2013randomized} demonstrated, algorithms optimized for undirected graphs may require substantial modifications to handle directed graphs, particularly in complex networks where edge directionality affects matching feasibility. Similarly, hypergraphs introduce a new level of complexity as they involve edges that connect multiple vertices \cite{Berge1989Hypergraphs}, making traditional algorithms ineffective without significant adaptations. These cases highlight the need for specialized algorithms or heuristics that account for unique structural properties to avoid performance bottlenecks.

\subsection{Generalization}
Many algorithms for maximum matching, such as the Hopcroft-Karp algorithm \cite{karzanov1973maximum} and Edmonds' Blossom algorithm \cite{edmonds1965paths}, are tailored for specific types of graphs (e.g., bipartite or undirected). Consequently, these algorithms may not generalize well to more complex structures like non-bipartite or directed hypergraphs. This limitation restricts their usability in applications requiring flexible solutions across varied graph structures. The lack of a universal algorithm applicable to all graph types often necessitates hybrid or multi-phase approaches, combining multiple methods to accommodate broader graph structures \cite{avis1983greedy}. As noted by Gabow and Tarjan \cite{gabow1991faster}, developing algorithms that can generalize effectively across different types of graphs remains an open research area.

\subsection{Limitations in Maximum Matching for d-Partite Graphs}
While significant advances have been made in solving maximum matching for bipartite graphs, extending these results to \textit{d-partite graphs} (where \(d > 2\)) remains challenging. Unlike bipartite graphs, for which polynomial-time algorithms such as the Hopcroft-Karp \cite{hopcroft1973n} and Hungarian methods \cite{kuhn1955hungarian} are effective, the maximum matching problem for d-partite graphs is known to be \textit{NP-hard} \cite{karp1972reducibility}. This difficulty arises because d-partite matching is closely
