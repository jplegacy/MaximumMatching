We begin by establishing the necessary foundations in graph theory, introducing essential concepts like simple graphs, directed versus undirected graphs, adjacency, and subgraphs \ref{sec:graph_theory}. These basic building blocks help form a solid understanding of how graphs are structured and provide the groundwork for more advanced discussions on maximum matching.

As we delve deeper into the topic, we examine various algorithms for maximum matching \ref{sec:existing_algorithms} and their respective runtime complexities. These algorithms are the heart of solving the matching problem, and we explore different approaches depending on the type of graph involved. Bipartite graphs \ref{sec:bipartite_algorithms}, for example, allow for specialized algorithms such as the Edmonds-Karp Algorithm \ref{subsec:edmonds_karp}, while more complex graph structures like d-partite graphs \ref{sec:dpartite_algorithms} require deeper considerations.

The book also covers related problems in graph theory \ref{sec:problem_variants} that have strong connections to maximum matching. Concepts like perfect matching \ref{sec:perfect_matching}, minimum vertex cover \ref{sec:min_vertex_cover}, and maximum flow \ref{sec:max_flow}
play important roles in understanding different matching scenarios. These problems, while distinct, often share underlying principles and solutions that enhance our comprehension of matchings.

In cases where finding exact solutions is computationally expensive, we explore approximation methods and heuristic algorithms \ref{sec:approximation_algorithms}. One such technique is Ant Colony Optimization (ACO) \ref{subsec:aco}, a powerful metaheuristic inspired by natural processes. This method is beneficial for approaching the maximum matching problem in complex and large-scale graphs, where an optimal solution may be too costly to compute.

Finally, we implemented the theory by testing various algorithms through races \ref{sec:testing_evaluation} to determine their efficiency on different graph types. These races helped evaluate performance under varying conditions, such as graph structure and size.

Testing was thorough, using both random and fixed test cases. The use of a visualizer helped verify solutions visually, while automated graph generation streamlined the process. However, automating NP-complete problems like d-partite graph generation posed challenges. Test files followed a consistent format, ensuring reproducibility.

Performance was assessed using metrics such as accuracy, total runtime, and a density-completeness metric. Solvers were evaluated in races based on these criteria, revealing key insights about their strengths and weaknesses. The races showed how algorithms performed in different scenarios, with some excelling in sparse graphs while others struggled with dense configurations. These results highlighted the trade-offs in efficiency and scalability across various solver types.
